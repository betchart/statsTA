\documentclass[letterpaper,11pt]{article}
\pdfoutput=1

\usepackage[margin=2cm]{geometry}

\title{Demonstration of technique failure}

\begin{document}
\maketitle

Consider a two dimensional distribution of four bins, where one axis
is a selection observable with bins (selected) and (unselected), and
the other is an observable for which the asymmetry is of interest,
with bins $(+)$ and $(-)$.  The selection contains $N_A$ events and
has asymmetry $\alpha_A$, while the unselected events number $N_B$ and
may have a different asymmetry $\alpha_B$.  The counts in each bin of
the model are given by
\[N_x^{\pm} = N_x(1\pm\alpha_x)/2,\]
and the total asymmetry is given by
\[\alpha = \frac{N_A\alpha_A + N_B\alpha_B}{N_A+N_B}.\]
Note that $\alpha_A\ne\alpha_B$ implies that the selection
efficiencies for the bins $(+)$ and $(-)$ are unequal.

Suppose we want to measure the asymmetry of a second distribution
which has an antisymmetric component proportional to that of the first
distribution, with factor $F$,
\[D_x^{\pm} = N_x(1\pm F\alpha_x)/2.\]
The total asymmetry would be $F\alpha$. Following the template
strategy we have so far employed, the weights which symmetrize and
antisymmetrize the original distribution after integrating over the
selection axis would be
\[W^\pm_{\mathrm{symm}} = (1\pm\alpha)^{-1},\]
\[W^\pm_{\mathrm{anti}} = \pm\alpha(1\pm\alpha)^{-1}.\]
The templates for the selection would be
\[T^{\pm}_{\mathrm{symm}} = \frac{N_A}{2}\frac{1\pm\alpha_A}{1\pm\alpha},\]
\[T^{\pm}_{\mathrm{anti}} = \pm\alpha\frac{N_A}{2}\frac{1\pm\alpha_A}{1\pm\alpha},\]
and the parametrized template model would be
\[T^{\pm}(f) = T^{\pm}_{\mathrm{symm}} + f\cdot T^\pm_{\mathrm{anti}} = \frac{N_A}{2}\frac{1\pm\alpha_A}{1\pm\alpha}(1\pm f\cdot\alpha).\]
Our expectation has been that $T^\pm(f) = D^\pm_A$ when $f=F$, but
that is clearly only the case if $F=f=1$ or $\alpha=\alpha_A$.  The
former condition is useless for a measurement, while the latter
condition is generally unmet, and is particularly untrue in the cases
of our subselections at extremes of $\mathrm{t\bar{t}}$ system
rapidity and mass.

A similar technique which would meet our former expectations for the
template model would derive the weights and subsequent templates from
only the selected distribution.  I am considering the possibilities
for implementing this option.

\vspace{20pt}

\noindent
Burt
\end{document}
