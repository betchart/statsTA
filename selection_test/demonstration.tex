\documentclass[letterpaper,11pt]{article}
\pdfoutput=1

\usepackage[margin=2cm]{geometry}

\title{Demonstration of technique failures}

\begin{document}
\maketitle

Consider a two dimensional distribution of four bins, where one axis
is a selection observable with bins (selected) and (unselected), and
the other is an observable for which the asymmetry is of interest,
with bins $(+)$ and $(-)$.  The selection contains $N_A$ events and
has asymmetry $\alpha_A$, while the unselected events number $N_B$ and
may have a different asymmetry $\alpha_B$.  The counts in each bin of
the model are given by
\[N_x^{\pm} = N_x(1\pm\alpha_x)/2,\]
and the total asymmetry is given by
\[\alpha = \frac{N_A\alpha_A + N_B\alpha_B}{N_A+N_B}.\]
Note that $\alpha_A\ne\alpha_B$ implies that the selection
efficiencies for the bins $(+)$ and $(-)$ are unequal.

Suppose we want to measure the asymmetry of a second distribution
which has an antisymmetric component proportional to that of the
modeled distribution, with factor $F$,
\[D_x^{\pm} = N_x(1\pm F\alpha_x)/2.\]
The total asymmetry would be $F\alpha$. 


\section{Template Strategy}
Following the template strategy we have so far employed, the weights
which symmetrize and antisymmetrize the original distribution after
integrating over the selection axis would be
\[W^\pm_{\mathrm{symm}} = (1\pm\alpha)^{-1},\]
\[W^\pm_{\mathrm{anti}} = \pm\alpha(1\pm\alpha)^{-1}.\]
The templates for the selection would be
\[T^{\pm}_{\mathrm{symm}} = \frac{N_A}{2}\frac{1\pm\alpha_A}{1\pm\alpha},\]
\[T^{\pm}_{\mathrm{anti}} = \pm\alpha\frac{N_A}{2}\frac{1\pm\alpha_A}{1\pm\alpha},\]
and the parametrized template model would be
\[T^{\pm}(f) = T^{\pm}_{\mathrm{symm}} + f\cdot T^\pm_{\mathrm{anti}} = \frac{N_A}{2}\frac{1\pm\alpha_A}{1\pm\alpha}(1\pm f\cdot\alpha).\]
Our expectation has been that $T^\pm(f) = D^\pm_A$ when $f=F$, but
that is clearly only the case if $F=f=1$ or $\alpha=\alpha_A$.  The
former condition is useless for a measurement, while the latter
condition is generally unmet, and is particularly untrue in the cases
of our subselections at extremes of $\mathrm{t\bar{t}}$ system
rapidity and mass.

\section{Unfolding Strategy}
The ratios $R^{\pm}=N^{\pm}_A/N^{\pm}_B$ given by the model are used
to calculate the corresponding unselected bins
\[U^{\pm}_B = D^{\pm}_A/R^\pm_m = D^\pm_B\left(\frac{1\pm\alpha_A}{1\pm\alpha_B}\right)\]
The expectation that $U^\pm_B=D^\pm_B$ is not met, leading to a
mistake in calculating a corresponding total asymmetry
\begin{eqnarray*}
  \alpha_{\mathrm{unfold}} &=& \frac{D^+_A+U^+_B - D^-_A - U^-_B}{D^+_A+U^+_A+D^-_A+U^-_B}\\
  &=& F \cdot \left(\frac{N_A\alpha_A + N_B\left[\alpha_A\left(\frac{1-\alpha_A\alpha_B}{1-\alpha_A^2}\right) + \frac{1}{F}\left(\frac{\alpha_B-\alpha_A}{1-\alpha_A^2}\right)\right]}{N_A+N_B\left[\left(\frac{1-\alpha_A\alpha_B}{1-\alpha_A^2}\right)+F\alpha_A\left(\frac{\alpha_B-\alpha_A}{1-\alpha_A^2}\right)\right]}\right)
\end{eqnarray*}
As an example of the significant measurement errors which can be
introduced by this flawed calculation, consider the plausible case
$\alpha_A=0\ne\alpha_B$.  We would find
\[\alpha_{\mathrm{unfold}} = \alpha \ne F\alpha.\]

\end{document}
