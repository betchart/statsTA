\documentclass{article}

\newcommand{\lik}{\mathcal{L}}
\newcommand{\interest}{\vec{x}}
\newcommand{\nuis}{\vec{y}}

\usepackage[margin=1.0in]{geometry}
\begin{document}

Given parameters of interest $\interest$ and nuisance parameters
$\nuis$, the probability to observe counts $\{N_{ij}^{\mu};N_{ij}^e\}$
in the muon and electron channels is
\[P 
= P\left(N_i^{\mu};N_i^e | y;\nuis\right) =
P_{\mu}\left(N_i^{\mu}|\interest;\nuis\right)P_e\left(N_i^e|\interest;\nuis\right)\]
If the values of the nuisance parameters are constrained by prior
probability $P_{cst}$, then the likelihood of the parameters given the
observations is
\begin{equation}
  \lik\left(\interest;\nuis|N_i^{\mu};N_i^e\right) = P_{cst}(\nuis)P_{\mu}(N_i^\mu|\interest;\nuis)P_e(N_i^e|\interest;\nuis)
\end{equation}


\begin{equation}
  P_\ell\left(N_i^\ell|\interest;\nuis\right) = \prod_i \mathrm{Pois}\left(N_i^\ell|\lambda_i^\ell\right)
\end{equation}
\begin{equation}
  \lambda_i^\ell = L\sum_j\sigma_j\epsilon_j^\ell\rho_{ji}^\ell
\end{equation}
where $L$ is the luminosity, $\sigma_j$ is the cross section of the
$j^{th}$ process, $\epsilon^\ell_j$ is the $\ell$ (muon or electron)
selection efficiency for the $j^{th}$ process, and $\rho^\ell_{ij}$
are the $i$ bins of the post-selection probability density
distribution of the $j^{th}$ process.  The distinct processes of
interest consist of 4 sources of $t\bar{t}$ production
($gg,qg,q\bar{q},\bar{q}g$), $W+$jets, multijet, single top, and
Drell-Yann.

We introduce the four parameters of interest $\{\delta_k\}$ and the
six nuisance parameters $\{\delta_L;\delta_j\}$ by allowing
\begin{eqnarray}
  L &=& \hat{L}(1+\delta_L) \\
  \sigma_j &=& \sigma_j^{mc}(1+\delta_j),\qquad j\notin \{k\to t\bar{t}\}, \quad k\in\{gg,qg,q\bar{q},\bar{q}g\}\\
  \sigma_{k\to t\bar{t}} &=& \sigma_{t\bar{t}}^{mc}(1+\delta_{t\bar{t}}) f_k\\
  f_k &=& f_k^{mc}(1+\delta_k)
\end{eqnarray}
The parameters of interest are constrained by the two equations
\begin{equation}
  \sum_k \delta_kf_k^{mc} = 0
\end{equation}
\begin{equation}
  D_{gg}D_{q\bar{q}} = D_{qg}D_{\bar{q}g}, \qquad D_k = (1+\delta_k)
\end{equation}
so they can be written as only two independent parameters,
$(\delta_{gg},\delta_{qg})$.  The other two are then
\begin{eqnarray}
  \delta_{\bar{q}g} &=& \left(\delta_{gg}f_{gg}^{mc}+\delta_{qg}f_{qg}^{mc}+\delta_{q\bar{q}}f_{q\bar{q}}^{mc}\right)/f_{\bar{q}g}^{mc}\\
  \delta_{q\bar{q}} &=& \left[f_{\bar{q}g}^{mc}(\delta_{qg}-\delta_{gg}) + D_{qg}(\delta_{qg}f_{qg}^{mc}+\delta_{gg}f_{gg}^{mc})\right] / \left(f_{\bar{q}g}^{mc}D_{gg} + f_{q\bar{q}}^{mc}D_{qg}\right)
\end{eqnarray}
The nuisance parameters are constrained by the prior probability
\begin{equation}
  P_{cst}(\delta_L;\delta_j) = \mathrm{Norm}(\delta_L/\omega_L)\prod_j\mathrm{Norm}(\delta_j/\omega_j)
\end{equation}
In summary, the inputs to the model are
\begin{center}
  \begin{tabular}{lrl}
    1&$\hat{L}$ & Luminosity prior best estimate \\
    1&$\omega_L$ & Luminosity prior uncertainty\\
    5&$\sigma_j^{mc}$ & Cross section prior best estimates\\
    5&$\omega_j$ & Cross section prior uncertainties\\
    4&$f^{mc}_k$ & $t\bar{t}$ Production fractions prior estimates\\
    16&$\epsilon_j^\ell$ & Efficiency of selection for $j$ sample in $\ell$ channel\\
    16$\sum_i1$&$\rho_{ji}^\ell$ & Probability density in $\sum_i1$ bins after selection for $j$ sample in $\ell$ channel\\
    2$\sum_i1$&$N_i^\ell$ & Observations in $\sum_i1$ bins in $\ell$ channel.
  \end{tabular}
\end{center}
The independent parameters of interest are $\vec{x} =
\{\delta_{gg},\delta_{qg}\}$, and the nuisance parameters ar $\vec{y}
= \{\delta_L;\delta_j\}$, with $j$ in $t\bar{t}$, $W+$jets, multijet,
Drell-Yann, and single top production.

\end{document}
